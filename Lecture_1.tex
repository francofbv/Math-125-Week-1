\documentclass{article}

\usepackage[english]{babel}
\usepackage[utf8]{inputenc}
\usepackage{amsmath,amssymb}
\usepackage{parskip}
\usepackage{graphicx}

% Margins
\usepackage[top=2.5cm, left=3cm, right=3cm, bottom=4.0cm]{geometry}
% Colour table cells
\usepackage[table]{xcolor}

% Get larger line spacing in table
\newcommand{\tablespace}{\\[1.25mm]}
\newcommand\Tstrut{\rule{0pt}{2.6ex}}         % = `top' strut
\newcommand\tstrut{\rule{0pt}{2.0ex}}         % = `top' strut
\newcommand\Bstrut{\rule[-0.9ex]{0pt}{0pt}}   % = `bottom' strut

%%%%%%%%%%%%%%%%%
%     Title     %
%%%%%%%%%%%%%%%%%
\title{Lecture 1: Anti Derivatives}
\author{Franco Vidal \\ Math 125}
\date{\today}

\begin{document}
\maketitle
\maketitle
\section*{Understanding the antiderivative}
\begin{center}
$\frac{dy}{dx} x^6 = 6x^5$
\end{center}
\begin{itemize}
\item The anti derivative is doing the above process backwards.
\item $x^6$ is an antiderivative of $6x^5$
\item an anti derivative of a function $f(x)$ on an interval is a function $F(x)$ for $F'(x) = f(x)$
\item hence, we say $x^6$ is an antiderivative of $6x^5$
\end{itemize}
\section*{How to find the anti derivative}
Method: guess and check \\
ex:
\begin{itemize}
    \item $f(x) = x^3$
    \begin{itemize}
        \item guess: $F(x) = x^4$
        \item check: $F'(x) = 4x^3$
    \end{itemize}
\end{itemize}
The above example shows that the two solutions are fairly close with the exception of the constant present in the seconds one
\\
Remember:
\begin{itemize}
    \item $\frac{d}{dx}(C \cdot g(x)) = C \frac{d}{dx}(g(x))$
\end{itemize}
For the previous example, set $C = \frac{1}{4}$
\begin{itemize}
    \item guess: $F(x) = \frac{1}{4}x^4$
    \item check: $F'(x) = \frac{1}{4}(4x^3) = x^3$
\end{itemize}
Another example:
\begin{itemize}
    \item $f(x) = xe^{x^2}$
    \begin{itemize}
        \item idea: work chain rule backwards
        \begin{itemize}
            \item guess: $F(x) = e^{x^2}$
            \item check: $F'(x) = e^{x^2} \cdot 2x = 2xe^{x^2}$
            \begin{itemize}
                \item This is almost correct but not quite
            \end{itemize}
            \item guess: $F(x) = \frac{1}{2}e^{x^2}$
            \item check: $F'(x) = \frac{1}{2}(e^{x^2} \cdot 2x) = xe^{x^2}$
            \begin{itemize}
                \item $F(x) = \frac{1}{2}e^{x^2}$ is an antiderivative of $f(x)=xe^{x^2}$
            \end{itemize}
        \end{itemize}
    \end{itemize}
\end{itemize}
Example:
\begin{itemize}
    \item $f(x) = x^3$
    \begin{itemize}
        \item try $F(x) = \frac{1}{4}x^4 + \pi$
        \item $F'(x) = \frac{1}{4}(4x^3+0) = x^3$
        \item hence, $F(x) = \frac{1}{4}x^4 + \pi$ is also an antiderivative of $f(x) = 3$
    \end{itemize}
\item there are more than one antiderivative for any given function
\end{itemize}
\section*{General antiderivatives}
\begin{itemize}
    \item if $f(x)$ is one antiderivative of $f(x)$ on an interval, then $F(x) + C$ is the general antiderivative of $f(x)$ on that interval
    \item new notation: $\int_{}^{} f(x)  \,dx $ is called the indefinite integral of f(x), it's the general antiderivative of f(x)
    \begin{itemize}
        \item it's a set of functions of $x$
    \end{itemize}
\end{itemize}
Example: guess and check
\begin{itemize}
    \item $\int_{}^{} (3x+1)dx = -\frac{1}{3}cos(3x+1) + C$
    \item guess: $F(x) = cos(3x+1)$
    \item check: $F'(x) = -sin(3x+1) \cdot 3 = -3sin(3x+1)$
    \item guess: $-\frac{1}{3}cos(3x+1)$
    \item check: $\frac{d}{dx}(-\frac{1}{3}cos(3x+1)) = -\frac{1}{3}(-sin(3x+1) \cdot 3) = sin(3x+1)$
\end{itemize}
\end{document}